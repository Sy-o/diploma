\sectioncentered*{ПРИЛОЖЕНИЕ А}
\addcontentsline{toc}{section}{Приложение А Исходный код программного средства.}
\begin{center}
\vspace{-1em}
\textbf{ (обязательное)}

\textbf{Исходный код программного средства}
\end{center}


\begin{lstlisting}[language=C++, style=cplusplusstyle]
#pragma once
#include "SimulatorObject.h"
#include "BusPacket.h"
#include "Rank.h"
#include "Discharger.h"
#include "FaultController.h"
#include "Address.h"

using namespace DRAMSim;

namespace DRAMSim
{
	class MemoryController;
	class DRAMDevice : public SimulatorObject
	{
		friend FaultController;
	public:
		DRAMDevice(MemoryController *mc, std::string faultFilePath = "");
		~DRAMDevice();

	public:
		void receiveFromBus(BusPacket *packet);
		void attachRanks(vector<Rank> *ranks);
		void update() override;
		void step() override;

		void setRefreshWaitingFlag(int rank);
		bool getRefreshWaitingFlag(int rank);
		void powerDown(int rank);
		void powerUp(int rank);

		void dischargeCells(int cycleCount, int rank);

	public:
		uint16_t read(Address addr);
		void write(Address addr, uint16_t data);
		void invertBit(Address addr);

	private:
		uint16_t readBit(Address addr);

	private:
		vector<Rank> *ranks;
		Discharger discharger;
		FaultController faultController;
	};
}


#include "DRAMDevice.h"
#include "MemoryController.h"
#include "SystemConfiguration.h"
#include "Parsing.h"
#include <vector>

using namespace std;
using namespace DRAMSim;

DRAMDevice::DRAMDevice(MemoryController *mc, string faultFilePath)
:faultController(this)
{
	currentClockCycle = 0;
	ranks = new vector<Rank>();

	for (size_t i = 0; i<NUM_RANKS; i++)
	{
		Rank r = Rank();
		r.setId(i);
		r.attachMemoryController(mc);
		ranks->push_back(r);
	}
	discharger.Initialize();
	if (!faultFilePath.empty())
	{
		vector<Fault> faults = ParseCSV(faultFilePath);
		faultController.SetFaults(faults);
	}
}

DRAMDevice::~DRAMDevice()
{
	ranks->clear();
	delete(ranks);
}

void DRAMDevice::attachRanks(vector<Rank> *ranks)
{
	this->ranks = ranks;
}

void DRAMDevice::update()
{
	//updates the state of each of the objects
	// NOTE - do not change order
	for (size_t i = 0; i<NUM_RANKS; i++)
	{
		(*ranks)[i].update();
	}
}

void DRAMDevice::step()
{
	for (size_t i = 0; i<NUM_RANKS; i++)
	{
		(*ranks)[i].step();
	}
	SimulatorObject::step();
}

void DRAMDevice::receiveFromBus(BusPacket *packet)
{
	if (packet->busPacketType == DATA)
	{
		faultController.DoFaults(packet->address, packet->data->getData());
		(*ranks)[packet->address.rank].receiveFromBus(packet);
	}
	else
	{
		(*ranks)[packet->address.rank].receiveFromBus(packet);
	}
}

uint16_t DRAMDevice::read(Address addr)
{
	return (*ranks)[addr.rank].banks[addr.bank].read(addr);
}

uint16_t DRAMDevice::readBit(Address addr)
{
	return (read(addr) & (1 << addr.bit)) ? 1 : 0;
}

void DRAMDevice::write(Address addr, uint16_t data)
{	
	faultController.DoFaults(addr, data);
}

void DRAMDevice::invertBit(Address addr)
{
	uint16_t oldBitValue = readBit(addr);
	uint16_t newBitValue = oldBitValue ? 0 : 1;
	faultController.DoOperationOnBit(addr, newBitValue, oldBitValue);
}

void DRAMDevice::setRefreshWaitingFlag(int rank)
{
	(*ranks)[rank].refreshWaiting = true;
}

bool DRAMDevice::getRefreshWaitingFlag(int rank)
{
	return (*ranks)[rank].refreshWaiting;
}

void DRAMDevice::powerDown(int rank)
{
	(*ranks)[rank].powerDown();
}

void DRAMDevice::powerUp(int rank)
{
	(*ranks)[rank].powerUp();
}

void DRAMDevice::dischargeCells(int cycleCount, int rank)
{
	std::vector<int> addrList = discharger.GetRandomAddressList(cycleCount);
	(*ranks)[rank].dischargeCells(addrList);
}

#pragma once
#include "BusPacket.h"
#include "DRAMDevice.h"
#include "Address.h"
#include "Rank.h"

class SAODCController
{
public:
    SAODCController();
    ~SAODCController();

public:
	void SetDRAMDevice(DRAMDevice *dram);
    void UpdateRef(DRAMSim::BusPacket* packet);

	void UpdateRef(Address addr, uint16_t data);
	void UpdateTestSig(Address addr, uint16_t data);

	int GetSignaturesSum();

	bool ParityBitInZero(int signature);
    int CalculateTestAndCompare();

	void ClearTestSig();
	void SetRefSignature(int signature);
	int GetTestSignature();

private:
    // function calculates address sum of each changed bit in word
    int GetAddress(Address addr, uint16_t data);
	void AddParityBit(int &address);

private:
    int RefSignature;
    int TestSignature;
	DRAMDevice *dramDevice;
};

#include "SystemConfiguration.h"
#include "SAODCController.h"

using namespace DRAMSim;

SAODCController::SAODCController()
{
	RefSignature = TestSignature = 0;
	dramDevice = 0;
}

SAODCController::~SAODCController()
{

}

void SAODCController::SetDRAMDevice(DRAMDevice *dram)
{
	dramDevice = dram;
}

void SAODCController::UpdateRef(BusPacket* packet)
{
    if (packet->busPacketType == DATA)
    {
		uint16_t oldData = dramDevice->read(packet->address);
        uint16_t newData = packet->data->getData();
       if (oldData != newData)
            UpdateRef(packet->address, newData ^ oldData);
    }
}

void SAODCController::UpdateRef(Address addr, uint16_t data)
{
	RefSignature ^= GetAddress(addr, data);
}

void SAODCController::UpdateTestSig(Address addr, uint16_t data)
{
	TestSignature ^= GetAddress(addr, data);
}

int SAODCController::GetSignaturesSum()
{
	return TestSignature ^ RefSignature;
}

int SAODCController::GetAddress(Address addr, uint16_t data)
{
    //get 'bit address sum'
	if (!data) return 0;
    
	int bitSum = 0;
    int bitCount  = 0; // '1' bit count
    for (int i = 0; (unsigned)i < DEVICE_WIDTH; i++)
    {
        if (data & (1 << i))
        {
            bitSum ^= i;
            bitCount++;
        }
    }

    if (bitCount % 2 == 0)
		addr.Clear();
    
	addr.bit = bitSum;
	int address = addr.GetPhysical();

	if (bitCount % 2 != 0)
		AddParityBit(address);
	
	return address;
}

int SAODCController::CalculateTestAndCompare()
{
    TestSignature = 0;
	for (int r = 0; r < NUM_RANKS; r++)
	{
		for (int b = 0; b < NUM_BANKS; b++)
		{
			for (int row = 0; row < NUM_ROWS; row++)
			{
				for (int col = 0; col < NUM_COLS; col++)
				{
					Address addr(r, b, row, col);
					uint16_t data = dramDevice->read(addr);
					UpdateTestSig(addr, data);
				}
			}
		}
	}

	return GetSignaturesSum();
}

void SAODCController::ClearTestSig()
{
	TestSignature = 0;
}

void SAODCController::SetRefSignature(int signature)
{
	RefSignature = signature;
}

int SAODCController::GetTestSignature()
{
	return TestSignature;
}

void SAODCController::AddParityBit(int &address)
{
	address <<= 1;
	address |= 1;
}

bool SAODCController::ParityBitInZero(int signature)
{
	return (signature & 1) == 0;
}

#pragma once
#include "SimulatorObject.h"
#include "SAODCController.h"
#include "MarchTestController.h"
#include "BusPacket.h"
#include "AddressTranslator.h"
#include "DRAMDevice.h"
#include "Rank.h"

class RegenerationController : public DRAMSim::SimulatorObject
{
public:
	RegenerationController();
	~RegenerationController();

public:
	void Initialize(DRAMDevice* dram);
	void UpdateRefSignature(DRAMSim::BusPacket* packet);
	void SetMarchTest(string marchTest);
	void StartRefresh();
	void update();

private:
	int GetMarchTestType(string marchTest);
		
private:
	MarchTestController marchController;
	SAODCController saodcController;
    AddressTranslator addrTranslator;

	DRAMDevice* dramDevice;

    int lastError;

	int refreshEndCycle;
	int startMarchTestCycle;

	bool needTest;
	
};

#include "RegenerationController.h"
#include "SystemConfiguration.h"
#include "MarchDef.h"
#include <iostream>

RegenerationController::RegenerationController()
{
    currentClockCycle = 0;
    lastError = 0;
    dramDevice = 0;
}

RegenerationController::~RegenerationController()
{
}

void RegenerationController::Initialize(DRAMDevice* dram)
{
	dramDevice = dram;
    refreshEndCycle = 0;
	startMarchTestCycle = 0;
	saodcController.SetDRAMDevice(dram);
	marchController.Initialize(dram);
	needTest = false;
}

void RegenerationController::SetMarchTest(string marchTest)
{
	marchController.SetMarchTest(GetMarchTestType(marchTest));
}

int RegenerationController::GetMarchTestType(string marchTest)
{
	if (marchTest.compare("March C-") == 0)
		return MARCH_C_MINUS;
	if (marchTest.compare("March A") == 0)
		return MARCH_A;
	if (marchTest.compare("March B") == 0)
		return MARCH_B;
	if (marchTest.compare("March X") == 0)
		return MARCH_X;
	if (marchTest.compare("March Y") == 0)
		return MARCH_Y;
	if (marchTest.compare("MATS") == 0)
		return MATS;
	if (marchTest.compare("MATS+") == 0)
		return MATS_PLUS;
	if (marchTest.compare("MATS++") == 0)
		return MATS_PLUS_PLUS;
	return MATS;
}

void RegenerationController::StartRefresh()
{
	refreshEndCycle = currentClockCycle + tRFC - 1;
    int err = saodcController.CalculateTestAndCompare();
    if (err)
	{
        std::cout << "[Regeneration Controller] Detected error(s) in memory.\n      Signature Sum = " << err << "(" << addrTranslator.GetDescription(err>>1, true) << ")" << std::endl;
        if (lastError)
        {
            needTest = true;
            startMarchTestCycle = currentClockCycle + 1;
        }
        else
        {
			if (saodcController.ParityBitInZero(err))
			{
				//multiple errors
				needTest = true;
				startMarchTestCycle = currentClockCycle + 1;
			}
			else
			{
				//if it first time - let's decide that it was 1 error. Invert bit
				lastError = err;
				std::cout << "[Regeneration Controller] Try to restore the damaged cell." << std::endl;
				Address addr(err >> 1, true);
				dramDevice->invertBit(addr);
			}
        }		
	}
    else
    {
        lastError = 0;
    }
}

void RegenerationController::update()
{
	if (currentClockCycle >= refreshEndCycle)
		return;

	if (needTest)
	{
		if (startMarchTestCycle == currentClockCycle)
		{
            std::cout << "[Regeneration Controller] March Test Started." << std::endl;
            marchController.RunTest(saodcController.GetTestSignature());
		}
		else if (startMarchTestCycle < currentClockCycle)
		{
			marchController.Update();
		}

		if (marchController.TestCompleted())
		{
			if (!marchController.TestPassed())
			{
                std::cout << "[Regeneration Controller] March Test Failed." << std::endl;
                //do smth in case of errors
			}
            else
            {
                std::cout << "[Regeneration Controller] March Test Passed." << std::endl;
            }
			marchController.Reset();
			needTest = false;
		}
	}
}

void RegenerationController::UpdateRefSignature(DRAMSim::BusPacket* packet)
{
	saodcController.UpdateRef(packet);
}

#pragma once

#include <vector>
#include "DRAMDevice.h"
#include "AddressTranslator.h"
#include "SAODCController.h"
#include "MarchDef.h"
#include <vector>

class MarchTestController
{
	struct State
	{
		bool testStarted;
		bool testCompleted;
		int phase;
		bool testPassed;

		State() { Clear(); }

		void Clear()
		{
			testStarted = false;
			testCompleted = false;
			phase = 0;
			testPassed = false;
		}
	};

	struct MarchPhase
	{
		MarchDirection direction;
		std::vector<MarchOperation> elements;
	};

public:
	MarchTestController();
	~MarchTestController();

public:
	void Initialize(DRAMDevice* dram);
	void SetMarchTest(int marchTest);

	void RunTest(int refSignature);
	void Update();
	bool TestCompleted();
	bool TestPassed();
	void Reset();

private:
	void RunElement(int element, int address, uint16_t &buffer, uint16_t& oldValue);

private:
	AddressTranslator addrTranslator;
	SAODCController saodc;
	State state;
	std::vector<MarchPhase> phases;
	DRAMDevice* dramDevice;
	bool testPass;
};

#include "MarchTestController.h"
#include "SystemConfiguration.h"
#include <iostream>

MarchTestController::MarchTestController()
{
	dramDevice = 0;
}

MarchTestController::~MarchTestController()
{}

void MarchTestController::Initialize(DRAMDevice* dram)
{
	dramDevice = dram;
}

void MarchTestController::SetMarchTest(int marchTest)
{
	phases.clear();
	switch (marchTest)
	{
	case MARCH_C_MINUS:
	{
					phases.push_back({ MD_UP, { MO_RD, MO_WDC } });
					phases.push_back({ MD_UP, { MO_RDC, MO_WD } });
					phases.push_back({ MD_DOWN, { MO_RD, MO_WDC } });
					phases.push_back({ MD_DOWN, { MO_RDC, MO_WD } });
					phases.push_back({ MD_BOTH, { MO_RD } });
					break;
	}
	case MARCH_B:
	{
					phases.push_back({ MD_UP, { MO_RD, MO_WDC, MO_RDC, MO_WD, MO_RD, MO_WDC } });
					phases.push_back({ MD_UP, { MO_RDC, MO_WD, MO_WDC } });
					phases.push_back({ MD_DOWN, { MO_RDC, MO_WD, MO_WDC, MO_WD } });
					phases.push_back({ MD_DOWN, { MO_RD, MO_WDC, MO_WD } });
					break;
	}
	case MARCH_A:
	{
					phases.push_back({ MD_UP, { MO_RD, MO_WDC, MO_WD, MO_WDC } });
					phases.push_back({ MD_UP, { MO_RDC, MO_WD, MO_WDC } });
					phases.push_back({ MD_DOWN, { MO_RDC, MO_WD, MO_WDC, MO_WD } });
					phases.push_back({ MD_DOWN, { MO_RD, MO_WDC, MO_WD } });
					break;
	}
	case MARCH_X:
	{
					phases.push_back({ MD_UP, { MO_RD, MO_WDC } });
					phases.push_back({ MD_DOWN, { MO_RDC, MO_WD } });
					phases.push_back({ MD_BOTH, { MO_RD } });
					break;
	}
	case MARCH_Y:
	{
					phases.push_back({ MD_UP, { MO_RD, MO_WDC, MO_RDC } });
					phases.push_back({ MD_DOWN, { MO_RDC, MO_WD, MO_RD } });
					phases.push_back({ MD_BOTH, { MO_RD } });
					break;
	}
	case MATS:
	{
					phases.push_back({ MD_UP, { MO_RD, MO_WDC } });
					phases.push_back({ MD_DOWN, { MO_RDC } });
					break;
	}
	case MATS_PLUS:
	{
					phases.push_back({ MD_UP, { MO_RD, MO_WDC } });
					phases.push_back({ MD_DOWN, { MO_RDC, MO_WD } });
					break;
	}
	case MATS_PLUS_PLUS:
	{
					phases.push_back({ MD_UP, { MO_RD, MO_WDC } });
					phases.push_back({ MD_DOWN, { MO_RDC, MO_WD, MO_RD } });
					break;
	}
	default:
		break;
	}
}

void MarchTestController::RunTest(int refSignature)
{
	Reset();
	state.testStarted = true;
	saodc.SetRefSignature(refSignature);
}

void MarchTestController::Update()
{
	if (!state.testStarted)
		return;

    saodc.ClearTestSig();

	MarchPhase& phase = phases[state.phase];

	int addrStart = 0;
	int counter = 0;

	int bottom = 0;
	int top = NUM_BANKS * NUM_COLS * NUM_ROWS - 1;

	if (phase.direction == MD_UP || phase.direction == MD_BOTH)
	{
		addrStart = bottom;
		counter = 1;
	}
	else if (phase.direction == MD_DOWN)
	{
		addrStart = top;
		counter = -1;
	}

	for (int addr = addrStart; addr >= bottom && addr <= top; addr += counter)
	{
		uint16_t buffer = 0;
		uint16_t old = 0;
		for (auto& el : phase.elements)
		{
			RunElement(el, addr, buffer, old);
		}
	}

	int err = saodc.GetSignaturesSum();
	if (err)
	{
		state.testPassed = false;
        cout << "[MarchTest] Errors are detected while running phase!\n      Signature sum is " << err << "(" << addrTranslator.GetDescription(err>>1, true) << ")" << endl;
	}

	state.phase++;
	if (state.phase == phases.size())
	{
		state.testCompleted = true;
		state.testStarted = false;
	}
}

void MarchTestController::RunElement(int element, int address, uint16_t &buffer, uint16_t& oldValue)
{
	Address addr(address, false);
	uint16_t data = 0;

	switch (element)
	{
	case MO_RD:
		data = dramDevice->read(addr);
		buffer = data;
		break;
	case MO_RDC:
		data = dramDevice->read(addr);
		buffer = ~data;
		break;
	case MO_WD:
		dramDevice->write(addr, buffer);
		break;
	case MO_WDC:
		dramDevice->write(addr, ~buffer);
		break;
	}

	//convertData
	if (element == MO_RD || element == MO_RDC)
	{
		saodc.UpdateTestSig(addr, data ^ oldValue);
		oldValue = data;
	}
}

bool MarchTestController::TestCompleted()
{
	return state.testCompleted;
}

bool MarchTestController::TestPassed()
{
	return state.testPassed;
}

void MarchTestController::Reset()
{
	state.testStarted = false;
	state.testPassed = true;
	state.testCompleted = false;
	state.phase = 0;
	saodc.ClearTestSig();
}

#pragma once

#include <vector>

enum MarchDirection
{
    MD_UP,
    MD_DOWN,
    MD_BOTH
};

enum MarchOperation
{
    MO_RD,
    MO_RDC,
    MO_WD,
    MO_WDC
};

enum MarchTest
{
    MARCH_C_MINUS,
    MARCH_B,
	MARCH_A,
	MARCH_Y,
	MARCH_X,
	MATS,
	MATS_PLUS,
	MATS_PLUS_PLUS
};

#pragma once

enum FaultType
{
	NONE = 0,
	SAF,
	TF,
	CFin,
	CFid,
	CFst,
	AF
};

struct Fault
{
	FaultType type;
	int victimValue;
	int agressorValue;
	int agressorAddress;
	int victimAddress;
};

#pragma once
#include "FaultDef.h"
#include "Address.h"
#include <stdint.h>
#include <vector>

namespace DRAMSim
{
	class DRAMDevice;
}

class FaultController
{

public:
	FaultController(DRAMSim::DRAMDevice* dram);
	~FaultController();

public:
	void SetFaults(std::vector<Fault>& faults);
	bool IsFaulty(int address);
	bool IsAgressor(int address);
	void DoFaults(Address addr, uint16_t data);
	void DoOperationOnBit(Address addr, uint16_t newVal, uint16_t oldVal);

	Fault GetCellAttributes(int address, bool isAgressor);

private:
	void WriteBit(Address address, int val);
	void InitSAFs();

private: 
	DRAMSim::DRAMDevice* dramDevice;
	AddressTranslator translator;
	std::vector<Fault> faultyCells;
};

#include "FaultController.h"
#include "DRAMDevice.h"
#include <algorithm>
using namespace std;

FaultController::FaultController(DRAMDevice* dram) : dramDevice(dram)
{
}

FaultController::~FaultController()
{
}

void FaultController::SetFaults(std::vector<Fault>& faults)
{
	faultyCells = faults;
	auto it = find_if(faultyCells.begin(), faultyCells.end(), [](Fault f){return f.type == SAF; });
	if (it != faultyCells.end())
	{
		InitSAFs();
	}
}

void FaultController::InitSAFs()
{
	for (auto& f : faultyCells)
	{
		if (f.type == SAF)
		{
			Address addr(f.victimAddress, true);
			(*dramDevice->ranks)[addr.rank].banks[addr.bank].writeBit(addr, f.victimValue == 1);
		}
	}
}

bool FaultController::IsFaulty(int address)
{
	auto it = find_if(faultyCells.begin(), faultyCells.end(), 
		[address](Fault f){return f.victimAddress == address && f.type != AF; });
	return it != faultyCells.end();
}

bool FaultController::IsAgressor(int address)
{
	auto it = find_if(faultyCells.begin(), faultyCells.end(), 
		[address](Fault f){return f.agressorAddress == address; });
	return it != faultyCells.end();
}

void FaultController::DoFaults(Address addr, uint16_t data)
{
	uint16_t oldData = dramDevice->read(addr);
	uint16_t bitMask = oldData ^ data;
	
	for (unsigned i = 0; i < DEVICE_WIDTH; i++)
	{
		if (bitMask & (1 << i))
		{
			addr.bit = i;
			DoOperationOnBit(addr, data & (1 << i) ? 1 : 0, oldData & (1 << i) ? 1 : 0);
		}
	}
}

void FaultController::DoOperationOnBit(Address address, uint16_t newVal, uint16_t oldVal)
{
	int addr = address.GetPhysical();
	if (IsFaulty(addr))
	{
		Fault fault = GetCellAttributes(addr, false);
		switch (fault.type)
		{
		case SAF: break;
		case TF:
		{
				   if (oldVal != fault.victimValue)
				   {
					   WriteBit(address, newVal);
				   }
				   break;
		}
		case CFin:
		{
					 WriteBit(address, newVal);
				     break;
		}			
		case CFid:
		{
					 WriteBit(address, newVal);
					 break;
		}
		case CFst:
		{
					 if (newVal != fault.victimValue)
						 WriteBit(address, newVal);
					 else
					 {
						 Address agrAdr(fault.agressorAddress, true);
						 if (dramDevice->readBit(agrAdr) == fault.agressorValue)
							 WriteBit(address, newVal);
					 }
					 break;
		}
		default: break;
		}
	}
	else
	{
		if (IsAgressor(addr))
		{
			Fault fault = GetCellAttributes(addr, true);			
			switch (fault.type)
			{
			case CFin:
			{
						 if (newVal == fault.agressorValue && newVal != oldVal)
						 {
							 Address vicAdr(fault.victimAddress, true);
							 (*dramDevice->ranks)[vicAdr.rank].banks[vicAdr.bank].invertBit(vicAdr);
						 }
						 break;
			}
			case CFid:
			{
						 if (newVal == fault.agressorValue && newVal != oldVal)
						 {
							 Address vicAdr(fault.victimAddress, true);
							 WriteBit(vicAdr, fault.victimValue);
						 }
						 break;
			}
			case AF:
			{
					   Address vicAdr(fault.victimAddress, true);
					   WriteBit(vicAdr, newVal);
					   break;
			}
			default :
				break;
			}
			WriteBit(address, newVal);
		}
		else
		{
			WriteBit(address, newVal);
		}
	}	
}

Fault FaultController::GetCellAttributes(int address, bool isAgressor)
{
	vector<Fault>::iterator it;
	if (isAgressor)
	{
		it = find_if(faultyCells.begin(), faultyCells.end(), [address](Fault f){return f.agressorAddress == address; });
	}
	else
	{
		it = find_if(faultyCells.begin(), faultyCells.end(), [address](Fault f){return f.victimAddress == address && f.type != AF; });
	}
	if (it != faultyCells.end())
		return *it;
	else return Fault();
}

void FaultController::WriteBit(Address address, int val)
{
	(*dramDevice->ranks)[address.rank].banks[address.bank].writeBit(address, val == 1);
}

#pragma once
#include <random>
#include <vector>

using namespace std;
class default_random_engine;
class exponential_distribution;

class Discharger
{
public:
    Discharger();
    ~Discharger();

public:
    void Initialize();
    vector<int> GetRandomAddressList(int curCycle);

private:
    std::random_device rd;
    std::mt19937* gen;
    std::uniform_int_distribution<>* uniformDist;
};

#include "Discharger.h"
#include "SystemConfiguration.h"
#include <math.h>

using namespace std;

Discharger::Discharger() : gen(0), uniformDist(0)
{
    
}

Discharger::~Discharger()
{
    delete uniformDist;
}

void Discharger::Initialize()
{
    gen = new std::mt19937(rd());
    int totalBits = NUM_RANKS * NUM_BANKS * NUM_ROWS * NUM_COLS * DEVICE_WIDTH;
    uniformDist = new std::uniform_int_distribution<>(0, totalBits - 1);
}

vector<int> Discharger::GetRandomAddressList(int curCycle)
{
    vector<int> addrList;
    double number = exp(sqrt(curCycle)/5.2);
    int addrCount = (int)number;
    
    for (int i = 0; i < addrCount; i++)
    {
        addrList.push_back((*uniformDist)(*gen));
    }
    return addrList;
}

#pragma once
#include <string>

class AddressTranslator
{
public:
    AddressTranslator();
    ~AddressTranslator();

	void Init();

    void Translate(int address, int& rank, int& bank, int& row,  int& col);
    void Translate(int address, int& rank, int& bank, int& row, int& col, int& bit);

	int TranslateToAddr(int rank, int bank, int row, int col, int bit = -1);

    std::string GetDescription(int address, bool includeBit);


private:
    int GetValue(int width, int& address);
	void SetValue(int width, int value, int& address);

private:
	int RankWidth;
    int BankWidth;
    int RowWidth;
    int ColWidth;
    int CellWidth;
};

#include "AddressTranslator.h"
#include "SystemConfiguration.h"
#include <sstream>

using namespace DRAMSim;

AddressTranslator::AddressTranslator()
{
	Init();
}

AddressTranslator::~AddressTranslator()
{}

void AddressTranslator::Init()
{
	RankWidth = dramsim_log2(NUM_RANKS);
	BankWidth = dramsim_log2(NUM_BANKS);
	RowWidth = dramsim_log2(NUM_ROWS);
	ColWidth = dramsim_log2(NUM_COLS);
	CellWidth = dramsim_log2(DEVICE_WIDTH);
}

int AddressTranslator::GetValue(int width, int& address)
{
    int temp = address;
    temp >>= width;
    temp <<= width;
    int n = temp ^ address;
    address >>= width;
    return n;
}

void AddressTranslator::SetValue(int width, int value, int& address)
{
	address <<= width;
	address |= value;	
}

void AddressTranslator::Translate(int address, int& rank, int& bank, int& row, int& col)
{
    col = GetValue(ColWidth, address);
    row = GetValue(RowWidth, address);
    bank = GetValue(BankWidth, address);
	rank = GetValue(RankWidth, address);
}

void AddressTranslator::Translate(int address, int& rank, int& bank, int& row, int& col, int& bit)
{
    bit = GetValue(CellWidth, address);
    col = GetValue(ColWidth, address);
    row = GetValue(RowWidth, address);
    bank = GetValue(BankWidth, address);
	rank = GetValue(RankWidth, address);
}

int AddressTranslator::TranslateToAddr(int rank, int bank, int row, int col, int bit)
{
	int address = 0;
	SetValue(RankWidth, rank, address);
	SetValue(BankWidth, bank, address);
	SetValue(RowWidth, row, address);
	SetValue(ColWidth, col, address);
	if (bit >= 0)
		SetValue(CellWidth, bit, address);
	return address;
}

std::string AddressTranslator::GetDescription(int address, bool includeBit)
{
    std::ostringstream stream;
    int r = 0, b = 0, row = 0, col = 0, bit = 0;
    if (includeBit)
        Translate(address, r, b, row, col, bit);
    else
        Translate(address, r, b, row, col);

    stream << "r:" << r << ", b:" << b << ", row:" << row << ", col:" << col;
    if (includeBit)
        stream << ", bit:" << bit;
   return stream.str();    
}

#pragma once
#include "AddressTranslator.h"

struct Address
{
public:
	Address(int r = 0, int b = 0, int row = 0, int col = 0, int bit = 0);
	Address(int address, bool full);

	void Clear();
	int GetPhysical();

	friend bool operator==(const Address& a, const Address& b)
	{
		return a.rank == b.rank &&
			   a.bank == b.bank &&
			   a.row == b.row &&
			   a.col == b.col &&
			   a.bit == b.bit;	
	}

public:
	int rank;
	int bank;
	int row;
	int col;
	int bit;

public:
	static void InitTranslator();
	static AddressTranslator translator;
};

#include "Address.h"
#include "SystemConfiguration.h"

AddressTranslator Address::translator;

Address::Address(int r, int b, int row, int col, int bit)
: rank(r), bank(b), row(row), col(col), bit(bit)
{
}

Address::Address(int address, bool full)
{
	Clear();
	if (full)
		translator.Translate(address, rank, bank, row, col, bit);
	else
		translator.Translate(address, rank, bank, row, col);	
}

void Address::Clear()
{
	rank = bank = row = col = bit = 0;
}

int Address::GetPhysical()
{
	return translator.TranslateToAddr(rank, bank, row, col, bit);
}

void Address::InitTranslator()
{
	translator.Init();
}

FaultType GetFaultTypeFromString(string type)
{
	if (type == "SAF") return SAF;
	if (type == "TF") return TF;
	if (type == "CFin") return CFin;
	if (type == "CFid") return CFid;
	if (type == "CFst") return CFst;
	if (type == "AF") return AF;
	return NONE;
}

vector<Fault> ParseCSV(string filePath)
{
	vector<Fault> faults;
	ifstream file(filePath);
	while (file.good())
	{
		string line;
		getline(file, line);

		Fault* fault = 0;
		int paramCount = 0;

		stringstream strstr(line);
		string value = "";

		while (getline(strstr, value, '\t'))
		{
			if (value == "SAF" || value == "TF" || value == "CFid" || value == "CFin" || value == "CFst" || value == "AF")
			{
				if (GetFaultTypeFromString(value) != NONE)
				{
					faults.push_back(Fault());
					fault = &faults.back();
					fault->type = GetFaultTypeFromString(value);
				}
			}
			else
			{
				if (fault)
				{
					paramCount++;
					switch (fault->type)
					{
					case SAF:
					case TF:
						if (paramCount == 1)
						{
							istringstream iss(value.substr(2));
							iss >> hex >> fault->victimAddress;
						}
						if (paramCount == 2)
						{
							istringstream iss(value);
							iss >> dec >> fault->victimValue;
						}
						break;
					case CFid:
					case CFst:
						if (paramCount == 1 || paramCount == 3)
						{
							istringstream iss(value.substr(2));
							if (paramCount == 1)
								iss >> hex >> fault->victimAddress;
							else
								iss >> hex >> fault->agressorAddress;
						}
						if (paramCount == 2 || paramCount == 4)
						{
							istringstream iss(value);
							if (paramCount == 2)
								iss >> dec >> fault->victimValue;
							else
								iss >> dec >> fault->agressorValue;
						}
						break;
					case CFin:
						if (paramCount == 1 || paramCount == 3)
						{
							istringstream iss(value.substr(2));
							if (paramCount == 1)
								iss >> hex >> fault->victimAddress;
							else
								iss >> hex >> fault->agressorAddress;
						}
						if (paramCount == 4)
						{
							istringstream iss(value);
							iss >> dec >> fault->agressorValue;
						}
						break;
					case AF:
						if (paramCount == 1 || paramCount == 3)
						{
							istringstream iss(value.substr(2));
							if (paramCount == 1)
								iss >> hex >> fault->victimAddress;
							else
								iss >> hex >> fault->agressorAddress;
						}
					}
				}
			}
		}
	}
	return faults;
}
\end{lstlisting}
