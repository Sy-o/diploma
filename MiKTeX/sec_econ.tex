\newcommand{\byr}{Br}

\section{Технико-экономическое обоснование разработки ПС}

% Begin Calculations

\FPeval{\totalProgramSize}{15150}
\FPeval{\totalProgramSizeCorrected}{6550}

\FPeval{\normativeManDays}{172} %Tn

\FPeval{\additionalComplexity}{0.12} %Ksl
\FPeval{\complexityFactor}{clip(1 + \additionalComplexity)}

\FPeval{\stdModuleUsageFactor}{0.7} %Kt
\FPeval{\originalityFactor}{0.7} %Kn

\FPeval{\adjustedManDaysExact}{clip( \normativeManDays * \complexityFactor * \stdModuleUsageFactor * \originalityFactor )}
\FPround{\adjustedManDays}{\adjustedManDaysExact}{0}

\FPeval{\daysInYear}{366}
\FPeval{\redLettersDaysInYear}{6}
\FPeval{\weekendDaysInYear}{105}
\FPeval{\vocationDaysInYear}{24}
\FPeval{\workingDaysInYear}{ clip( \daysInYear - \redLettersDaysInYear - \weekendDaysInYear - \vocationDaysInYear ) }

\FPeval{\developmentTimeMonths}{3}
\FPeval{\developmentTimeYearsExact}{clip(\developmentTimeMonths / 12)}
\FPround{\developmentTimeYears}{\developmentTimeYearsExact}{2}
\FPeval{\requiredNumberOfProgrammersExact}{ clip( \adjustedManDays / (\developmentTimeYears * \workingDaysInYear) + 0.5 ) }

% тут должно получаться 2 ))
\FPtrunc{\requiredNumberOfProgrammers}{\requiredNumberOfProgrammersExact}{0}

\FPeval{\tariffRateFirst}{298000} %Tm1
\FPeval{\tariffFactorFst}{3.04}
\FPeval{\tariffFactorSnd}{3.48}


\FPeval{\employmentFstExact}{clip( \adjustedManDays / \requiredNumberOfProgrammers )}
\FPtrunc{\employmentFst}{\employmentFstExact}{0}

\FPeval{\employmentSnd}{clip(\adjustedManDays - \employmentFst)}


\FPeval{\workingHoursInMonth}{160} %Fr
\FPeval{\salaryPerHourFstExact}{clip( \tariffRateFirst * \tariffFactorFst / \workingHoursInMonth )}
\FPeval{\salaryPerHourSndExact}{clip( \tariffRateFirst * \tariffFactorSnd / \workingHoursInMonth )}
\FPround{\salaryPerHourFst}{\salaryPerHourFstExact}{0}
\FPround{\salaryPerHourSnd}{\salaryPerHourSndExact}{0}

\FPeval{\bonusRate}{1.5}
\FPeval{\workingHoursInDay}{8}
\FPeval{\totalSalaryExact}{clip( \workingHoursInDay * \bonusRate * ( \salaryPerHourFst * \employmentFst + \salaryPerHourSnd * \employmentSnd ) )}
\FPround{\totalSalary}{\totalSalaryExact}{0}

\FPeval{\additionalSalaryNormative}{20}

\FPeval{\additionalSalaryExact}{clip( \totalSalary * \additionalSalaryNormative / 100 )}
\FPround{\additionalSalary}{\additionalSalaryExact}{0}

\FPeval{\socialNeedsNormative}{0.5}
\FPeval{\socialProtectionNormative}{34}
\FPeval{\socialProtectionFund}{ clip(\socialNeedsNormative + \socialProtectionNormative) }

\FPeval{\socialProtectionCostExact}{clip( (\totalSalary + \additionalSalary) * \socialProtectionFund / 100 )}
\FPround{\socialProtectionCost}{\socialProtectionCostExact}{0}

\FPeval{\taxWorkProtNormative}{4}
\FPeval{\taxWorkProtCostExact}{clip( (\totalSalary + \additionalSalary) * \taxWorkProtNormative / 100 )}
\FPround{\taxWorkProtCost}{\taxWorkProtCostExact}{0}
\FPeval{\taxWorkProtCost}{0} % это считать не нужно, зануляем чтобы не менять формулы

\FPeval{\stuffNormative}{3}
\FPeval{\stuffCostExact}{clip( \totalSalary * \stuffNormative / 100 )}
\FPround{\stuffCost}{\stuffCostExact}

\FPeval{\timeToDebugCodeNormative}{15}
%\FPeval{\reducingTimeToDebugFactor}{0.3} %?
\FPeval{\adjustedTimeToDebugCodeNormative}{\timeToDebugCodeNormative}

\FPeval{\oneHourMachineTimeCost}{5000}

\FPeval{\machineTimeCostExact}{ clip( \oneHourMachineTimeCost * \totalProgramSizeCorrected / 100 * \adjustedTimeToDebugCodeNormative ) }
\FPround{\machineTimeCost}{\machineTimeCostExact}{0}

\FPeval{\businessTripNormative}{15}
\FPeval{\businessTripCostExact}{ clip( \totalSalary * \businessTripNormative / 100 ) }
\FPround{\businessTripCost}{\businessTripCostExact}{0}

\FPeval{\otherCostNormative}{20}
\FPeval{\otherCostExact}{clip( \totalSalary * \otherCostNormative / 100 )}
\FPround{\otherCost}{\otherCostExact}{0}

\FPeval{\overheadCostNormative}{100}
\FPeval{\overallCostExact}{clip( \totalSalary * \overheadCostNormative / 100 )}
\FPround{\overheadCost}{\overallCostExact}{0}

\FPeval{\overallCost}{clip( \totalSalary + \additionalSalary + \socialProtectionCost + \taxWorkProtCost + \stuffCost + \machineTimeCost + \businessTripCost + \otherCost + \overheadCost ) }

\FPeval{\supportNormative}{30}
\FPeval{\softwareSupportCostExact}{clip( \overallCost * \supportNormative / 100 )}
\FPround{\softwareSupportCost}{\softwareSupportCostExact}{0}


\FPeval{\baseCost}{ clip( \overallCost + \softwareSupportCost ) }

\FPeval{\profitability}{35}
\FPeval{\incomeExact}{clip( \baseCost / 100 * \profitability )}
\FPround{\income}{\incomeExact}{0}

\FPeval{\estimatedPrice}{clip( \income + \baseCost )}

\FPeval{\localRepubTaxNormative}{3.9}
\FPeval{\localRepubTaxExact}{clip( \estimatedPrice * \localRepubTaxNormative / (100 - \localRepubTaxNormative) )}
\FPround{\localRepubTax}{\localRepubTaxExact}{0}
%\FPeval{\localRepubTax}{0}

\FPeval{\ndsNormative}{20}
\FPeval{\ndsExact}{clip( (\estimatedPrice + \localRepubTax) / 100 * \ndsNormative )}
\FPround{\nds}{\ndsExact}{0}


\FPeval{\sellingPrice}{clip( \estimatedPrice + \localRepubTax + \nds )}

\FPeval{\taxForIncome}{18}
\FPeval{\incomeWithTaxesExact}{clip(\income * (1 - \taxForIncome / 100))}
\FPround{\incomeWithTaxes}{\incomeWithTaxesExact}{0}

% End Calculations

Целью дипломного проекта является создание программного средства для верификации алгоритмов тестирования оперативных запоминающих устройств.
Данное программное средство позволяет облегчить верификацию ОЗУ, не прибегая к тестированию реальных микросистем. Основными достоинствами программного средства являются: симуляция реальных процессов ОЗУ, что позволяет обеспечить максимальную схожесть с реальными системами на чипе, существенное увелечение удобства процесса запуска и верификации тестов динамической оперативной памяти, актуальность.

В данном разделе рассмотрим экономическую эффективность программного средства. Программный комплекс относится ко 2-ой группе сложности. Категория новизны продукта - «В».
Для оценки экономической эффективности разработанного программного средства проводится расчет цены и прибыли от продажи одной системы(программы).

Расчеты выполнены на основе методического пособия ~\cite{palicyn_2006}.

\subsection{Расчёт сметы затрат и цены программного продукта}

Целесообразность создания коммерческого ПО требует проведения предварительной экономической оценки и расчета экономического эффекта.
Экономический эффект у разработчика ПО зависит от объёма инвестиций в разработку проекта, цены на готовый программный продукт и количества проданных копий, и проявляется в виде роста чистой прибыли.

Исходные данные для разрабатываемого проекта указаны в таблице~\ref{table:econ:initial_data}.

\begin{table}[!ht]
\caption{Исходные данные}
\label{table:econ:initial_data}
  \centering
  \begin{tabular}{| >{\raggedright}m{0.62\textwidth}
                  | >{\centering}m{0.17\textwidth}
                  | >{\centering\arraybackslash}m{0.13\textwidth}|}
    \hline
    {\begin{center}
      Наименование
    \end{center} } & Условное обозначение & Значение \\
    \hline
    Категория сложности & & 2 \\

    \hline
    Коэффициент сложности, ед. & $ \text{К}_\text{с} $ & \num{\complexityFactor} \\

    \hline
    Степень использования при разработке стандартных модулей, ед. & $ \text{К}_\text{т} $ & \num{\stdModuleUsageFactor} \\

    \hline
    Коэффициент новизны, ед. & $ \text{К}_\text{н} $ & \num{\originalityFactor} \\

    \hline
    Годовой эффективный фонд времени, дн. & $ \text{Ф}_\text{эф} $ & \num{\workingDaysInYear} \\

    \hline
    Продолжительность рабочего дня, ч. & $ \text{Т}_\text{ч} $ & \num{\workingHoursInDay} \\

    \hline
    Месячная тарифная ставка первого разряда, \byr{} & $ \text{Т}_{\text{м}_{1}}$ & \num{\tariffRateFirst} \\

    \hline
    Коэффициент премирования, ед. & $ \text{К} $ & \num{\bonusRate} \\

    \hline
    Норматив дополнительной заработной платы, ед. & $ \text{Н}_\text{д} $ & \num{\additionalSalaryNormative} \\

    \hline
    Норматив отчислений в ФСЗН и обязательное страхование, $\%$ & $ \text{Н}_\text{сз} $ & \num{\socialProtectionFund} \\

    \hline
    Норматив командировочных расходов, $\%$ & $ \text{Н}_\text{к} $ & \num{\businessTripNormative} \\

    \hline
    Норматив прочих затрат, $\%$ & $ \text{Н}_\text{пз} $ & \num{\otherCostNormative} \\

    \hline
    Норматив накладных расходов, $\%$ & $ \text{Н}_\text{рн} $ & \num{\overheadCostNormative} \\

    \hline
    Прогнозируемый уровень рентабельности, $\%$ & $ \text{У}_\text{рп} $ & \num{\profitability} \\

    \hline
    Норматив НДС, $\%$ & $ \text{Н}_\text{дс} $ & \num{\ndsNormative} \\

    \hline
    Норматив налога на прибыль, $\%$ & $ \text{Н}_\text{п} $ & \num{\taxForIncome} \\

    \hline
    Норматив расхода материалов, $\%$ & $ \text{Н}_\text{мз} $ & \num{\stuffNormative} \\

    \hline
    Норматив расхода машинного времени, ч. & $ \text{Н}_\text{мв} $ & \num{\adjustedTimeToDebugCodeNormative} \\

    \hline
    Цена одного часа машинного времени, \byr{} & $ \text{Н}_\text{мв} $ & \num{\oneHourMachineTimeCost} \\

    \hline
    Норматив расходов на сопровождение и адаптацию ПО, $\%$ & $ \text{Н}_\text{рса} $ & \num{\supportNormative} \\
    \hline
  \end{tabular}
\end{table}

На основании сметы затрат и анализа рынка ПО определяется плановая отпускаемая цена.
Для составления сметы затрат на создание ПО необходима предварительная оценка трудоемкости ПО и его объёма.
Расчет объёма программного продукта (количества строк исходного кода) предполагает определение типа программного обеспечения, всестороннее техническое обоснование функций ПО и определение объёма каждой функций.
Согласно классификации типов программного обеспечения~\cite[с.~59,~приложение 1]{palicyn_2006}, разрабатываемое ПО с наименьшей ошибкой можно классифицировать как ПО методo"=ориентированных расчетов.


Общий объём программного продукта определяется исходя из количества и объёма функций, реализованных в программе:
\begin{equation}
  \label{eq:econ:total_program_size}
  V_{o} = \sum_{i = 1}^{n} V_{i} \text{\,,}
\end{equation}
\begin{explanation}
где & $ V_{i} $ & объём отдельной функции ПО, LoC; \\
    & $ n $ & общее число функций.
\end{explanation}

На стадии технико-экономического обоснования проекта рассчитать точный объём функций невозможно.
Вместо вычисления точного объёма функций применяются приблизительные оценки на основе данных по аналогичным проектам или по нормативам~\cite[с.~61,~приложение 2]{palicyn_2006}, которые приняты в организации.

\begin{table}[ht]
\caption{Перечень и объём функций программного модуля}
\label{table:econ:function_sizes}
\centering
  \begin{tabular}{| >{\centering}m{0.12\textwidth}
                  | >{\raggedright}m{0.40\textwidth}
                  | >{\centering}m{0.18\textwidth}
                  | >{\centering\arraybackslash}m{0.18\textwidth}|}

  \hline
         \multirow{2}{0.12\textwidth}[-0.5em]{\centering \No{} функции}
       & \multirow{2}{0.40\textwidth}[-0.55em]{\centering Наименование (содержание)}
       & \multicolumn{2}{c|}{\centering Объём функции, LoC} \tabularnewline

  \cline{3-4} &
       & { по каталогу ($ V_{i} $) }
       & { уточненный ($ V_{i}^{\text{у}} $) } \tabularnewline

  \hline
  101 & Организация ввода информации & \num{150} & \num{70} \tabularnewline

  \hline
  102 & Контроль, предварительная обработка и ввод информации & \num{450} & \num{300} \tabularnewline

  \hline
  111 & Управление вводом/выводом & \num{2400} & \num{1000} \tabularnewline

  \hline
  301 & Формирование последовательного файла & \num{290} & \num{250} \tabularnewline

  \hline
  305 & Обработка файлов & \num{420} & \num{350} \tabularnewline

  \hline
  501 & Монитор ПО (управление работой компонентов) & \num{740} & \num{700} \tabularnewline

  \hline
  506 & Обработка ошибочных и сбойных ситуаций & \num{410} & \num{400} \tabularnewline

  \hline
  507 & Обеспечение интерфейса между компонентами & \num{970} & \num{680} \tabularnewline

  \hline
  701 & Математическая статистика и прогнозирование & \num{9320} & \num{2800} \tabularnewline

  \hline

  % Уточенная оценка вычислялась с помощью R: (+ручной фикс)
  % set.seed(35)
  % locs <- c(100, 520, 2700, 520, 750, 1100, 430, 730, 460, 8370)
  % locs.which.corrected <- rbinom(length(locs), 1, 0.4)
  % locs.corrections <- rnorm(length(locs), mean = -0.25, sd=0.3)
  % locs.correction.factor <- 1 + locs.which.corrected * locs.corrections
  % locs.corrected <- signif(locs * locs.correction.factor, digits = 2)
  % locs.corrected
  % sum(locs)
  % sum(locs.corrected)

  Итог & & {\num{\totalProgramSize}} & {\num{\totalProgramSizeCorrected}} \tabularnewline

  \hline

  \end{tabular}
\end{table}

Перечень и объём функций программного модуля перечислен в таблице~\ref{table:econ:function_sizes}.
По приведенным данным уточненный объём некоторых функций изменился, и общий уточненный объём ПО $ V_{\text{у}} = \SI{\totalProgramSizeCorrected}{\text{LoC}} $.

\subsection{Расчёт нормативной трудоемкости}

На основании общего объема ПО определяется нормативная трудоемкость ($ \text{Т}_\text{н}$) с учетом сложности ПО. Для ПО 2-ой группы сложности, к которой относится разрабатываемый программный продукт, нормативная трудоемкость составит~$ \text{Т}_\text{н} = \SI{\normativeManDays}{\text{чел.} / \text{дн.}}  $

Нормативная трудоемкость служит основой для оценки общей трудоемкости~$ \text{Т}_\text{о} $.
Используем формулу (\ref{eq:econ:effort_common}) для оценки общей трудоемкости для небольших проектов:
\begin{equation}
  \label{eq:econ:effort_common}
  \text{Т}_\text{о} = \text{Т}_\text{н} \cdot
                      \text{К}_\text{с} \cdot
                      \text{К}_\text{т} \cdot
                      \text{К}_\text{н} \text{\,,}
\end{equation}
\begin{explanation}
где & $ \text{К}_\text{с} $ & коэффициент, учитывающий сложность ПО; \\
    & $ \text{К}_\text{т} $ & поправочный коэффициент, учитывающий степень использования при разработке стандартных модулей; \\
    & $ \text{К}_\text{н} $ & коэффициент, учитывающий степень новизны ПО.
\end{explanation}

Дополнительные затраты труда на разработку ПО учитываются через коэффициент сложности, который вычисляется по формуле
\begin{equation}
\label{eq:econ:complexity_coeff}
  \text{К}_{\text{с}} = 1 + \sum_{i = 1}^n \text{К}_{i} \text{\,,}
\end{equation}
\begin{explanation}
где & $ \text{К}_{i} $ & коэффициент, соответствующий степени повышения сложности ПО за счет конкретной характеристики; \\
    & $ n $ & количество учитываемых характеристик.
\end{explanation}

Наличие двух характеристик сложности позволяет~\cite[c.~66, приложение~4, таблица~П.4.2]{palicyn_2006} вычислить коэффициент сложности
\begin{equation}
\label{eq:econ:complexity_coeff_calc}
  \text{К}_{\text{с}} = \num{1} + \num{\additionalComplexity} = \num{\complexityFactor} \text{\,.}
\end{equation}

Разрабатываемое ПО использует стандартные компоненты. Согласно справочным данным~\cite[c.~68,~приложение~4, таблица~П.4.5]{palicyn_2006} коэффициент использования стандартных модулей для разрабатываемого приложения $ \text{К}_\text{т} = \num{\stdModuleUsageFactor} $.
Разрабатываемое ПО не является новым, существуют аналогичные более зрелые разработки у различных компаний и университетов по всему миру.
Влияние степени новизны на трудоемкость создания ПО определяется коэффициентом новизны~---~$ \text{К}_\text{н} $.
Согласно справочным данным~\cite[c.~67, приложение~4, таблица~П.4.4]{palicyn_2006} для разрабатываемого ПО $ \text{К}_\text{н} = \num{\originalityFactor} $.
Подставив приведенные выше коэффициенты для разрабатываемого ПО в формулу~(\ref{eq:econ:effort_common}) получим общую трудоемкость разработки
\begin{equation}
  \label{eq:econ:effort_common_calc}
  \text{Т}_\text{о} = \num{\normativeManDays} \times \num{\complexityFactor} \times \num{\stdModuleUsageFactor} \times \num{\originalityFactor} \approx \SI{\adjustedManDays}{\text{чел.}/\text{дн.}}
\end{equation}

На основе общей трудоемкости и требуемых сроков реализации проекта вычисляется плановое количество исполнителей.
Численность исполнителей проекта рассчитывается по формуле:
\begin{equation}
  \label{eq:econ:num_of_programmers}
  \text{Ч}_\text{р} = \frac{\text{Т}_\text{о}}{\text{Т}_\text{р} \cdot \text{Ф}_\text{эф}} \text{\,,}
\end{equation}
\begin{explanation}
где & $ \text{Т}_\text{о} $ & общая трудоемкость разработки проекта, $ \text{чел.}/\text{дн.} $; \\
    & $ \text{Ф}_\text{эф} $ & эффективный фонд времени работы одного работника в течение года, дн.; \\
    & $ \text{Т}_\text{р} $ & срок разработки проекта, лет.
\end{explanation}

Эффективный фонд времени работы одного разработчика вычисляется по формуле
\begin{equation}
  \label{eq:econ:effective_time_per_programmer}
  \text{Ф}_\text{эф} =
    \text{Д}_\text{г} -
    \text{Д}_\text{п} -
    \text{Д}_\text{в} -
    \text{Д}_\text{о} \text{\,,}
\end{equation}
\begin{explanation}
где & $ \text{Д}_\text{г} $ & количество дней в году, дн.; \\
    & $ \text{Д}_\text{п} $ & количество праздничных дней в году, не совпадающих с выходными днями, дн.; \\
    & $ \text{Д}_\text{в} $ & количество выходных дней в году, дн.; \\
    & $ \text{Д}_\text{п} $ & количество дней отпуска, дн.
\end{explanation}

Согласно данным, приведенным в производственном календаре для пятидневной рабочей недели в 2016 году для Беларуси~\cite{belcalendar_2016}, фонд рабочего времени составит
\begin{equation}
  \text{Ф}_\text{эф} = \num{\daysInYear} - \num{\redLettersDaysInYear} - \num{\weekendDaysInYear} - \num{\vocationDaysInYear} = \SI{\workingDaysInYear}{\text{дн.}}
\end{equation}

Учитывая срок разработки проекта $ \text{Т}_\text{р} = \SI{\developmentTimeMonths}{\text{мес.}} = \SI{\developmentTimeYears}{\text{года}} $, общую трудоемкость и фонд эффективного времени одного работника, вычисленные ранее, можем рассчитать численность исполнителей проекта
\begin{equation}
  \label{eq:econ:num_of_programmers_calc}
  \text{Ч}_\text{р} =
    \frac{\num{\adjustedManDays}}
         {\num{\developmentTimeYears} \times \num{\workingDaysInYear}}
    \approx \SI{\requiredNumberOfProgrammers}{\text{рабочих}}.
\end{equation}

Вычисленные оценки показывают, что для выполнения запланированного проекта в указанные сроки необходимо два рабочих.

\subsection{Расчёт основной заработной платы исполнителей}

Информация о работниках перечислена в таблице~\ref{table:econ:programmers}.
\begin{table}[ht]
  \caption{Работники, занятые в проекте}
  \label{table:econ:programmers}
  \begin{tabular}{| >{\centering}m{0.4\textwidth}
                  | >{\centering}m{0.15\textwidth}
                  | >{\centering}m{0.18\textwidth}
                  | >{\centering\arraybackslash}m{0.15\textwidth}|}
   \hline
   Исполнители & Разряд & Тарифный коэффициент & \mbox{Чел./дн.} занятости \\
   \hline
   Программист \Rmnum{1}-категории & $ \num{13} $ & $ \num{\tariffFactorFst} $ & $ \num{\employmentFst} $ \\
   \hline
   Ведущий программист & $ \num{15} $ & $ \num{\tariffFactorSnd} $ & $ \num{\employmentSnd} $ \\
   \hline
  \end{tabular}
\end{table}

Месячная тарифная ставка одного работника вычисляется по формуле
\begin{equation}
  \label{eq:econ:month_salary}
  \text{Т}_\text{ч} =
    \frac {\text{Т}_{\text{м}_{1}} \cdot \text{Т}_{\text{к}} }
          {\text{Ф}_{\text{р}} }  \text{\,,}
\end{equation}
\begin{explanation}
где & $ \text{Т}_{\text{м}_{1}} $ & месячная тарифная ставка 1-го разряда, \byr; \\
    & $ \text{Т}_{\text{к}} $ & тарифный коэффициент, соответствующий установленному тарифному разряду; \\
    & $ \text{Ф}_{\text{р}} $ & среднемесячная норма рабочего времени, час.
\end{explanation}




Подставив данные из таблицы~\ref{table:econ:programmers} в формулу~(\ref{eq:econ:month_salary}), приняв значение тарифной ставки 1-го разряда $ \text{Т}_{\text{м}_{1}} = \SI{\tariffRateFirst}{\text{\byr}} $ и среднемесячную норму рабочего времени $ \text{Ф}_{\text{р}} = \SI{\workingHoursInMonth}{\text{часов}} $ получаем
\begin{equation}
  \label{eq:econ:month_salary_calc1}
  \text{Т}_{\text{ч}}^{\text{прогр. \Rmnum{1}-разр.}} = \frac{ \num{\tariffRateFirst} \times \num{\tariffFactorFst} } { \num{\workingHoursInMonth} } = \SI{\salaryPerHourFst}{\text{\byr}/\text{час;}}
\end{equation}
\begin{equation}
  \label{eq:econ:month_salary_calc2}
  \text{Т}_{\text{ч}}^{\text{вед. прогр.}} = \frac{ \num{\tariffRateFirst} \times \num{\tariffFactorSnd} } { \num{\workingHoursInMonth} } = \SI{\salaryPerHourSnd}{\text{\byr}/\text{час.}}
\end{equation}

Основная заработная плата исполнителей на конкретное ПО рассчитывается по формуле
\begin{equation}
  \label{eq:econ:total_salary}
  \text{З}_{\text{о}} = \sum^{n}_{i = 1}
                        \text{Т}_{\text{ч}}^{i} \cdot
                        \text{Т}_{\text{ч}} \cdot
                        \text{Ф}_{\text{п}} \cdot
                        \text{К}
                          \text{\,,}
\end{equation}
\begin{explanation}
где & $ \text{Т}_{\text{ч}}^{i} $ & часовая тарифная ставка \mbox{$ i $-го} исполнителя, \byr$/$час; \\
    & $ \text{Т}_{\text{ч}} $ & количество часов работы в день, час; \\
    & $ \text{Ф}_{\text{п}} $ & плановый фонд рабочего времени \mbox{$ i $-го} исполнителя, дн.; \\
    & $ \text{К} $ & коэффициент премирования.
\end{explanation}

Подставив ранее вычисленные значения и данные из таблицы~\ref{table:econ:programmers} в формулу~(\ref{eq:econ:total_salary}) и приняв коэффициент премирования $ \text{К} = \num{\bonusRate} $ получим
\begin{equation}
  \label{eq:econ:total_salary_calc}
  \text{З}_{\text{о}} = (\salaryPerHourFst \times \num{\employmentFst} + \salaryPerHourSnd \times \num{\employmentSnd}) \times \num{\workingHoursInDay} \times \num{\bonusRate} = \SI{\totalSalary}{\text{\byr}} \text{\,.}
\end{equation}

Дополнительная заработная плата включает выплаты предусмотренные законодательством от труде и определяется по нормативу в процентах от основной заработной платы
\begin{equation}
  \label{eq:econ:additional_salary}
  \text{З}_{\text{д}} =
    \frac {\text{З}_{\text{о}} \cdot \text{Н}_{\text{д}}}
          {100\%} \text{\,,}
\end{equation}
\begin{explanation}
  где & $ \text{Н}_{\text{д}} $ & норматив дополнительной заработной платы, $ \% $.
\end{explanation}

Приняв норматив дополнительной заработной платы $ \text{Н}_{\text{д}} = \num{\additionalSalaryNormative\%} $ и подставив известные данные в формулу~(\ref{eq:econ:additional_salary}) получим
\begin{equation}
  \label{eq:econ:additional_salary_calc}
  \text{З}_{\text{д}} =
    \frac{\num{\totalSalary} \times 20\%}
         {100\%} \approx \SI{\additionalSalary}{\text{\byr}} \text{\,.}
\end{equation}

\begin{table}[!ht]
  \caption{Расчет себестоимости и отпускной цены ПО}
  \label{table:econ:calculation_cost_and_price}
  \begin{tabular}{| >{\raggedright}m{0.27\textwidth}
                  | >{\centering}m{0.16\textwidth}
                  | >{\centering}m{0.35\textwidth}
                  | >{\centering\arraybackslash}m{0.15\textwidth}|}
   \hline
     {\begin{center}
       Наименование статей 
    \end{center} } & Норматив & Методика расчета & \mbox{Значение,} руб. \\
   \hline
   Отчисления в фонд социальной защиты и обязательного страхования 
   & $ \text{Н}_{\text{сз}} = \num{\socialProtectionFund\%} $ 
   & $ \text{З}_{\text{сз}} = (\text{З}_{\text{о}} + \text{З}_{\text{д}}) \cdot \text{Н}_{\text{сз}} / {\num{100}} $
   & \num{\socialProtectionCost}\\
   \hline
   Материалы и комплектующие
   & $ \text{Н}_{\text{мз}} = \num{\stuffNormative\%} $
   & $\text{М} = { \text{З}_{\text{о}} \cdot \text{Н}_{\text{мз}} } / { \num{100} } $
   & \num{\stuffCost} \\
   \hline
   Машинное время 
   & $ \text{Н}_{\text{мв}} = \num{\timeToDebugCodeNormative} $
   & $ \text{Р}_{\text{м}} = \text{Ц}_{\text{м}} \cdot \text{V}_{\text{о}} / \num{100} \cdot \text{Н}_{\text{мв}} $
   $ \text{Ц}_{\text{м}} = \SI{\oneHourMachineTimeCost}{\text{\byr}} $ 
   & \num{\machineTimeCost} \\
   \hline
   Расходы на научные командировки
   & $ \text{Н}_{\text{к}} = \num{\businessTripNormative\%} $
   & $  \text{Р}_{\text{к}} = { \text{З}_{\text{о}} \cdot \text{Н}_{\text{к}} } / \num{100} $
   & \num{\businessTripCost} \\
   \hline
   Прочие прямые расходы
   & $ \text{Н}_{\text{пз}} = \num{\otherCostNormative\%} $
   & $  \text{П}_{\text{з}} = { \text{З}_{\text{о}} \cdot \text{Н}_{\text{пз}} } / \num{100} $
   & \num{\otherCost} \\
   \hline
   Накладные расходы
   & $ \text{Н}_{\text{рн}} = \num{\overheadCostNormative\%} $
   & $  \text{Р}_{\text{н}} = { \text{З}_{\text{о}} \cdot \text{Н}_{\text{рн}} } / \num{100} $
   & \num{\overheadCost}\\
   \hline
   Общая сумма расходов по смете
   & 
   & $  \text{С}_{\text{р}} = \text{З}_{\text{о}} + \text{З}_{\text{д}} + \text{З}_{\text{сз}} + \text{М} + \text{Р}_{\text{м}} + \text{Р}_{\text{к}} + \text{П}_{\text{з}} + \text{Р}_{\text{н}} $
   & \num{\overallCost}\\
   \hline
   Сопровождение и адаптация ПО
   & $ \text{Н}_{\text{рса}} = \num{\supportNormative\%} $
   & $  \text{Р}_{\text{са}} = {\text{С}_{\text{р}} \cdot \text{Н}_{\text{рса}} } / { \num{100} } $
   & \num{\softwareSupportCost} \\
   \hline
   Полная себестоимость ПО
   & 
   & $ \text{С}_{\text{п}} = \text{С}_{\text{р}} + \text{Р}_{\text{са}} $
   & \num{\baseCost} \\
   \hline
   Прогнозируемая прибыль
   & $ \text{У}_{\text{рп}} = \num{\profitability\%} $
   & $  \text{П}_{\text{с}} = { \text{С}_{\text{п}} \cdot \text{У}_{\text{рп}} } / \num{100} $
   & \num{\income} \\
   \hline 
   Прогнозируемая цена без налогов
   & 
   & $ \text{Ц}_{\text{п}} = \text{С}_{\text{п}} + \text{П}_{\text{с}}$
   & \num{\estimatedPrice} \\
   \hline
   Отчисления и налоги в местный и республиканский бюджеты
   & $ \text{Н}_{\text{мр}} = \num{\localRepubTaxNormative\%} $
   & $ \text{О}_{\text{мр}} = { \text{Ц}_{\text{п}} \cdot \text{Н}_{\text{мр}} } / { \num{100} - \text{Н}_{\text{мр}} } $
   & \num{\localRepubTax} \\
   \hline
   Налог на добавленную стоимость
   & $ \text{Н}_{\text{дс}} = \num{\ndsNormative\%} $
   & $ \text{НДС}_{\text{}} = { (\text{Ц}_{\text{п}} + \text{О}_{\text{мр}}) \cdot \text{Н}_{\text{дс}} } / \num{100} $
   & \num{\nds} \\
   \hline
   Прогнозируемая отпускная цена
   & 
   & $ \text{Ц}_{\text{о}} = \text{Ц}_{\text{п}} + \text{О}_{\text{мр}} + \text{НДС} $
   & \num{\sellingPrice} \\
   \hline
  \end{tabular}
\end{table}

\subsection{Расчёт экономической эффективности у разработчика}

Важная задача при выборе проекта для финансирования это расчет экономической эффективности проектов и выбор наиболее выгодного проекта.
Разрабатываемое ПО является заказным, т.е. разрабатывается для одного заказчика на заказ. На основании анализа рыночных условий и договоренности с заказчиком об отпускной цене прогнозируемая рентабельность проекта составит $ \text{У}_{\text{рп}} = \num{\profitability\%} $.

Чистую прибыль от реализации проекта можно рассчитать по формуле
\begin{equation}
  \label{eq:econ:income_with_taxes}
  \text{П}_\text{ч} =
    \text{П}_\text{c} \cdot
    \left(1 - \frac{ \text{Н}_\text{п} }{ \num{100\%} } \right) \text{\,,}
\end{equation}
\begin{explanation}
  где & $ \text{Н}_{\text{п}} $ & величина налога на прибыль,~$\%$.
\end{explanation}

Приняв значение налога на прибыль $ \text{Н}_{\text{н}} = \num{\taxForIncome\%} $ и подставив известные данные в формулу~(\ref{eq:econ:income_with_taxes}) получаем чистую прибыль
\begin{equation}
  \label{eq:econ:income_with_taxes_calc}
  \text{П}_\text{ч} =
    \num{\income} \times \left( 1 - \frac{\num{\taxForIncome\%}}{100\%} \right) = \SI{\incomeWithTaxes}{\text{\byr}} \text{\,.}
\end{equation}

Программное обеспечение разрабатывалось для одного заказчика в связи с этим экономическим эффектом разработчика будет являться чистая прибыль от реализации~$ \text{П}_\text{ч} $.
Рассчитанные данные приведены в таблице~\ref{table:econ:calculated_data}.

\begin{table}[!ht]
\caption{Рассчитанные данные}
\label{table:econ:calculated_data}
  \centering
  \begin{tabular}{| >{\raggedright}m{0.60\textwidth}
                  | >{\centering}m{0.17\textwidth}
                  | >{\centering\arraybackslash}m{0.15\textwidth}|}
    \hline
    {\begin{center}
      Наименование
    \end{center} } & Условное обозначение & Значение \\
    \hline
    Нормативная трудоемкость, чел.$/$дн. & $ \text{Т}_\text{н} $ & \num{\normativeManDays} \\

    \hline
    Общая трудоемкость разработки, чел.$/$дн. & $ \text{Т}_\text{о} $ & \num{\adjustedManDays} \\

    \hline
    Численность исполнителей, чел. & $ \text{Ч}_\text{р} $ & \num{\requiredNumberOfProgrammers} \\

    \hline
    Часовая тарифная ставка программиста \Rmnum{1}-разряда, \byr{}$/$ч. & $ \text{Т}_{\text{ч}}^{\text{прогр. \Rmnum{1}-разр.}} $ & \num{\salaryPerHourFst} \\

    \hline
    Часовая тарифная ставка ведущего программиста, \byr{}$/$ч. & $ \text{Т}_{\text{ч}}^{\text{вед. прогр.}} $ & \num{\salaryPerHourSnd} \\

    \hline
    Основная заработная плата, \byr{} & $ \text{З}_\text{о} $ & \num{\totalSalary} \\

    \hline
    Дополнительная заработная плата, \byr{} & $ \text{З}_\text{д}$ & \num{\additionalSalary} \\

    \hline
    Отчисления в фонд социальной защиты, \byr{} & $ \text{З}_\text{сз} $ & \num{\socialProtectionCost} \\

    \hline
    Затраты на материалы, \byr{} & $ \text{М} $ & \num{\stuffCost} \\

    \hline
    Расходы на машинное время, \byr{} & $ \text{Р}_\text{м} $ & \num{\machineTimeCost} \\

    \hline
    Расходы на командировки, \byr{} & $ \text{Р}_\text{к} $ & \num{\businessTripCost} \\

    \hline
    Прочие затраты, \byr{} & $ \text{П}_\text{з} $ & \num{\otherCost} \\

    \hline
    Накладные расходы, \byr{} & $ \text{Р}_\text{н} $ & \num{\overheadCost} \\

    \hline
    Общая сумма расходов по смете, \byr{} & $ \text{С}_\text{р} $ & \num{\overallCost} \\

    \hline
    Расходы на сопровождение и адаптацию, \byr{} & $ \text{Р}_\text{са} $ & \num{\softwareSupportCost} \\

    \hline
    Полная себестоимость, \byr{} & $ \text{С}_\text{п} $ & \num{\baseCost} \\

    \hline
    Прогнозируемая прибыль, \byr{} & $ \text{П}_\text{с} $ & \num{\income} \\

    \hline
    НДС, \byr{} & $ \text{НДС} $ & \num{\nds} \\

    \hline
    Прогнозируемая отпускная цена ПО, \byr{} & $ \text{Ц}_\text{о} $ & \num{\sellingPrice} \\

    \hline
    Чистая прибыль, \byr{} & $ \text{П}_\text{ч} $ & \num{\incomeWithTaxes} \\

    \hline
  \end{tabular}
\end{table}

\subsection{Выводы по технико-экономическому обоснованию}

Программное средство для верификации алгоритмов тестирования оперативных запоминающих устройств является выгодным программным продуктом.
Чистая прибыль от реализации ПС ($ \text{П}_\text{ч} $ \num{\incomeWithTaxes} рублей) остается организации-разработчику и представляет собой экономический эффект от создания нового программного средства.
Прогнозируемая отпускная цена ($\text{Ц}_\text{о}$) составляет \num{\sellingPrice} рублей. Таким образом, данная разработка является экономически целесообразной.
В итоге было произведено технико-экономическое обоснование разрабатываемого проекта, составлена смета затрат и рассчитана прогнозируемая прибыль, а также показана экономическая целесообразность разработки.

\hfill
\clearpage