\sectioncentered*{Введение}
\addcontentsline{toc}{section}{Введение}
\label{sec:intro}

Оперативные запоминающие устройства выполняют одну из важнейших функций в современных цифровых системах обработки и хранения информации. На протяжении последних лет наблюдается устойчивая тенденция по совершенствованию технологии производства цифровых устройств. Увеличение удельного веса запоминающих устройств по сравнению с операционными устройствами в вычислительных системах является следствием достижений современной микроэлектроники и вычислительной техники. 

Постоянное увеличение емкости и уменьшение технологических норм производства оперативной памяти приводит к значительному увеличению количества сбоев и отказов запоминающих устройств в процессе эксплуатации цифровой техники. По результатам исследований отказы ОЗУ составляют до 70\% от общего числа отказов вычислительных систем. Таким образом, обнаружение неисправных состояний ОЗУ является весьма важной и актуальной проблемой. Для повышения отказоустойчивости и надежности оперативных запоминающих устройств применяется ряд мер, позволяющих обнаружить, локализовать и исправить возникающие неисправности и ошибки. Традиционно эти проблемы решаются с применением маршевых тестов, эффективно обнаруживающих простейшие модели неисправностей ОЗУ. Но с неуклонным ростом емкости ОЗУ, которая превышает $10^9$ бит, а также существенное отличие физической структуры ОЗУ от их логической организации, делает задачу обнаружения неисправностей всё более сложной и трудоемкой. С каждым годом разрабатываются всё новые и новые тесты, описываются достаточно сложные модели неисправностей, вследствие чего крайне необходим аппарат для проверки тестирующих алгоритмов на различных  моделях неисправностей. Это улучшит качество и позволит достичь высоких показателей надежности и отказоустойчивости современных цифровых систем.

В данном дипломном проекте реализуются некоторые из известных алгоритмов тестирования оперативных запоминающих устройств. Моделируются наиболее распространенные функциональные неисправности ОЗУ, а также проверяется покрывающая способность неразрушающих маршевых тестов.
В результате получилась библиотека симулятора динамического ОЗУ, написанная на языке программирования С++, пригодная для решения практических задач в реальных перспективах.
