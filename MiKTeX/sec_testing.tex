\section{Тестирование приложения}
\label{sec:testing}
Программное средство разрабатывалось, как библиотека. Для использования симулятора необходима внешняя среда в качестве драйвера, который создаст объект класса MemorySystem и будет добавлять транзакции в очередь, а также отсчитывать циклы работы симулятора. Тестирование предполагает проверку работоспособности самой библиотеки в рамках драйвера. Поэтому перед запусками тестов-кейсов необходимо заранее проделать следующие шаги:
\begin{enumerate}
  \item создать консольное приложение на языке С++;
  \item подключить библиотеку в проект;
  \item создать объект класса MemorySystem;
  \item создать очередь для транзакций; 
  \item организовать цикл для отсчета тактов для симулятора.
\end{enumerate}

Тестирование предполагает наличие файлов конфигурации и файла инициализации для симулятора. Все значения, записываемые в память, не должны превышать максимальный объём хранимых данных в ячейке. 

Для оценки правильности работы программного средства было проведено тестирование. Тест-кейсы для проверки базовой функциональности программного средства представлены в таблицах \ref{sec:testing:configuration} и \ref{sec:testing:data_storing}.

\begin{longtable}[l]{| >{\raggedright}p{0.3\textwidth}
                     | >{\raggedright}p{0.3\textwidth}
                     | >{\raggedright\arraybackslash}p{0.3\textwidth}|}
  \caption{Тестирование корректной конфигурации памяти}
  \label{sec:testing:configuration} \tabularnewline

  \hline
       Название тест-кейса и его описание & Ожидаемый результат & Фактический результат результат \\
   \hline
   Верная конфигурация памяти \\ 
   1) Инициализировать объект MemorySystem входными параметрами: путь к ini-файлу, путь к файлу конфигурации памяти, объем памяти; \\ 
   2) запустить цикл работы симулятора.

   &
   1) В консоли отображается информация о запуске системы, количестве ранков и объеме симулируемой памяти; \\
   2) объем памяти и количество ранков совпадает со значениями в файле конфигурации памяти.

   & Тест пройден \\
   \hline
\end{longtable}

\begin{longtable}[l]{| >{\raggedright}p{0.3\textwidth}                     
                     | >{\raggedright}p{0.3\textwidth}
                     | >{\raggedright\arraybackslash}p{0.3\textwidth}|}
  \caption{Тестирование хранения данных в памяти}
  \label{sec:testing:data_storing} \tabularnewline

  \hline
      Название тест-кейса и его описание & Ожидаемый результат & Фактический результат результат \\
   \hline
   Хранение данных в памяти\\ 
   1) Инициализировать объект MemorySystem входными параметрами: путь к ini-файлу, путь к файлу конфигурации памяти, объем памяти; \\
   2) запустить цикл работы симулятора;\\
   3) добавить в очередь транзакций операцию записи значения по адресу;\\
   4) добавить в очередь транзакций операцию чтения по этому же адресу.

   &
   1) В консоли отображается информация о запуске системы и объеме симулируемой памяти;\\
   2) отображается информация о получении пакета с операцией записи;\\
   3) отображается информация о возвращаемом пакете с результатом чтения;\\
   4) результат чтения равен значению, записанному ранее операцией записи в память.

   &
   Тест пройден \\
   \hline
\end{longtable}

Тестирование процесса разрядки конденсаторов памяти в случае сдвига периода обновления памяти представлено в таблице \ref{sec:testing:degradation}.
\pagebreak

\begin{longtable}[p]{| >{\raggedright}p{0.3\textwidth}                     
                     | >{\raggedright}p{0.3\textwidth}
                     | >{\raggedright\arraybackslash}p{0.3\textwidth}|}
  \caption{Тестирование процесса деградации памяти при сдвинутом периоде обновления}
  \label{sec:testing:degradation} \tabularnewline

  \hline
      Название тест-кейса и его описание & Ожидаемый результат & Фактический результат результат \\
   \hline
   Деградация памяти\\ 
   1) Инициализировать объект MemorySystem входными параметрами: путь к ini-файлу, путь к файлу конфигурации памяти, объем памяти, количество циклов, на которое сдвинут период обновления памяти; \\
   2) добавить в очередь транзакций операцию записи значения, все биты которого находятся в состоянии логической единицы(максимально возможное значение, хранимое ячейкой);\\
   3) запустить цикл работы симулятора;\\
   4) через равные промежутки времени в цикле добавлять операции чтения по адресу, в который ранее было записано значение.

   &
   1) В консоли отображается информация о запуске системы и объеме симулируемой памяти;\\
   2) спустя период времени, после которого должен начаться период обновления памяти(в соответствии с файлом конфигурации) каждая новая операция чтения возвращает постепенно убывающее значение.

   &
   Тест пройден \\
   \hline
\end{longtable}

Следует протестировать поведение системы при наличии в памяти многократных ошибок. Для этого запускаются тесты из таблицы \ref{sec:testing:regeneration}.
\begin{longtable}[p]{| >{\raggedright}p{0.3\textwidth}                     
                     | >{\raggedright}p{0.3\textwidth}
                     | >{\raggedright\arraybackslash}p{0.3\textwidth}|}
  \caption{Тестирование процесса регенерации  памяти при сдвинутом периоде обновления}
  \label{sec:testing:regeneration} \tabularnewline

  \hline
      Название тест-кейса и его описание & Ожидаемый результат & Фактический результат результат \\
   \hline
   Регенерация памяти\\ 
   1) Инициализировать объект MemorySystem входными параметрами: путь к ini-файлу, путь к файлу конфигурации памяти, объем памяти, количество циклов, на которое сдвинут период обновления памяти; \\
   2) запустить цикл работы симулятора.

   &
   1) В консоли отображается информация о запуске системы и объеме симулируемой памяти;\\
   2) после наступления периода обновления появляется сообщение о несовпадении эталонной и тестовой сигнатур;\\
   3) появляется уведомление о запуске маршевого теста;\\
   4) маршевый тест пройден, неисправности не обнаружены.

   &
   Тест пройден \\
   \hline
\end{longtable}

В открытых источниках находится информация о покрывающей способности некоторых маршевых тестов, например MATS, MATS++ и других. На основе этой информации строятся тесты, для проверки правильности работы тестирующих алгоритмов ОЗУ. Тесты предполагают наличие файла неисправностей в формате csv. Общий алгоритм тестирования проверки работоспособности маршевых тестов един и сводится к таблице \ref{sec:testing:march_testing}.
\pagebreak

\begin{longtable}[p]{| >{\raggedright}p{0.3\textwidth}                     
                     | >{\raggedright}p{0.3\textwidth}
                     | >{\raggedright\arraybackslash}p{0.3\textwidth}|}
  \caption{Тестирование поведения системы при наличии в памяти неисправностей}
  \label{sec:testing:march_testing} \tabularnewline

  \hline
      Название тест-кейса и его описание & Ожидаемый результат & Фактический результат результат \\
   \hline
   Работоспособность маршевых тестов\\ 
   1) Инициализировать объект MemorySystem входными параметрами: путь к ini-файлу, путь к файлу конфигурации памяти, объем памяти, путь к файлу неисправностей; \\
   2) запустить цикл работы симулятора.

   &
   1) В консоли отображается информация о запуске системы и объеме симулируемой памяти;\\
   2) после наступления периода обновления появляется сообщение о несовпадении эталонной и тестовой сигнатур;\\
   3) появляется уведомление о запуске маршевого теста;\\
   4) маршевый тест не пройден, обнаружены неисправности.

   &
   Тест пройден \\
   \hline
\end{longtable}

Таким образом, результат тестирования подтверждает, что программное средство верификации алгоритмов тестирования оперативных запоминающих устройств функционирует в полном соответствии со спецификацией требований.