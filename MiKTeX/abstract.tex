\sectioncentered*{Реферат}
\thispagestyle{empty}

\begin{center}
Пояснительная записка 104 c., \totfig{}~рис., \tottab{}~табл., \toteq{}~формул и \totref{}~источников.\\
\MakeUppercase{ОЗУ, тестирующие алгоритмы, функциональные неисправности, верификация, адаптивный сигнатурный анализатор, встроенное самотестирование}
\end{center}

Объектом исследований является покрывающая способность тестирующих алгоритмов ОЗУ, а также возможность верифицировать любой тест на любой модели неисправности.
Цель работы -- разработка программного средства верификации алгоритмов тестирования на различных функциональных моделях неисправностей ОЗУ.
Создание данного программного средства обеспечит возможность улучшения качества разрабатываемых тестирующих алгоритмов, которые можно будет проверить на самых сложных моделях неисправностей.

Проведен анализ методов встроенного самотестирования ОЗУ, устройства работы динамических оперативных запоминающих устройств, современных тестирующих алгоритмов, а также видов функциональных моделей неисправностей ОЗУ.

Результатом разработки стала библиотека, написанная на языке программирования С++, которая содержит в себе классы симулятора динамической памяти. Данное программное средство легко внедрить в любую систему, отсутствие зависимостей от сторонних библиотек облегчает компиляцию под любой операционной системой. 

Предполагается использование библиотеки программистами, занимающимися проблемами создания и верифицирования тестирующих алгоритмов оперативных запоминающих устройств.

Разработанное программное средство является экономически эффективным, т.к. полностью оправдывает вложенные в разработку средства. 

\clearpage
